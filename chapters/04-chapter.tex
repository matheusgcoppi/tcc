% ----------------------------------------------------------
\chapter{TESTE OU AVALIAÇÃO} \label{cap:teste}
% ----------------------------------------------------------

Testes e avaliações tem o poder de enriquecer o trabalho acadêmico, fornecendo dados que permitirão ao leitor avaliar a qualidade da solução desenvolvida. Este capítulo pode apresentar, por exemplo, um teste de usabilidade com três seções: 4.1 PLANEJAMENTO, 4.2 EXECUÇÃO e 4.3 RESULTADOS.
Como outro exemplo, este capítulo pode apresentar um estudo de caso ou simulação com o uso da solução desenvolvida. Neste caso, uma possível estrutura de seções seria: 4.1 AMBIENTE DE ESTUDO, 4.2 IMPLANTAÇÃO, 4.3 RESULTADOS.
Para apresentação de dados ou estatísticas, utilize tabelas, lembrando que, diferente das ilustrações, as legendas das tabelas aparecem na parte superior.

% \begin{table}[htb]
% 	\ABNTEXfontereduzida
% 	\caption{\label{tab:Tab_1}Exemplo de tabela em Látex.}
% 	\begin{tabular}{@{}p{3.0cm}p{1.5cm}p{2cm}p{2.5cm}p{2.5cm}p{2.5cm}@{}}
% 		\toprule
% 		\textbf{Média concentração urbana} & \multicolumn{2}{l}{\textbf{População}} & \textbf{Produto Interno Bruto – PIB (bilhões R\$)} & \textbf{Número de empresas} & \textbf{Número de unidades locais} \\ \midrule
% 		\textbf{Nome}                      & \textbf{Total}   & \textbf{No Brasil}  &                                                   &                             & \\
% 		Ji-Paraná (RO)                     & 116 610          & 116 610             & 1,686                                             & 2 734                       & 3 082 \\
% 		Parintins (AM)                     & 102 033          & 102 033             & 0,675                                             & 634                         & 683 \\
% 		Boa Vista (RR)                     & 298 215          & 298 215             & 4,823                                             & 4 852                       & 5 187 \\
% 		Bragança (PA)                      & 113 227          & 113 227             & 0,452                                             & 654                         & 686 \\ \bottomrule
% 	\end{tabular}
% 	\fonte{Elaborado pelo autor deste trabalho (2018).}
% \end{table}