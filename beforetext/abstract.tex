\begin{resumo}[Abstract]
	\SingleSpacing
	\begin{otherlanguage*}{english}
		The increasing complexity of systems and the demand for scalability have driven the adoption of microservices architecture over the monolithic model. This approach promotes modularization, granular scalability, and development team autonomy, but also introduces operational and observability challenges. In this Undergraduate Thesis, a comparison is conducted between the communication protocols \textit{REST} and \textit{gRPC} in microservices-based architectures, using observability metrics such as latency (P95 and P99), throughput, and resource usage. The methodology involves implementing two equivalent services, instrumenting them with Prometheus, conducting load tests, and performing statistical analysis of the results. The objective is to identify the most efficient protocol in different usage scenarios. The expected results aim to provide technical insights to guide architectural decisions in distributed systems, promoting a balance between performance and flexibility.
		
		\textbf{Keywords}: microservices. \textit{REST}. \textit{gRPC}.
	\end{otherlanguage*}
\end{resumo}