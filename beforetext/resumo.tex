% resumo em português
\setlength{\absparsep}{18pt} % ajusta o espaçamento dos parágrafos do resumo
\begin{resumo}
	\SingleSpacing
	A crescente complexidade dos sistemas e a demanda por escalabilidade impulsionaram a adoção da arquitetura de microsserviços em detrimento do modelo monolítico. Essa abordagem favorece a modularização, a escalabilidade granular e a autonomia das equipes de desenvolvimento, mas também impõe desafios operacionais e de observabilidade. Neste Trabalho de Conclusão de Curso, realiza-se uma comparação entre os protocolos de comunicação \textit{REST} e \textit{gRPC} em arquiteturas baseadas em microsserviços, utilizando métricas de observabilidade como latência (P95 e P99), \textit{throughput} e uso de recursos. A metodologia envolve a implementação de dois serviços equivalentes, instrumentação com Prometheus, execução de testes de carga e análise estatística dos resultados. O objetivo é identificar o protocolo mais eficiente em diferentes cenários de uso. Os resultados esperados visam fornecer subsídios técnicos para orientar decisões arquiteturais em sistemas distribuídos, promovendo um equilíbrio entre desempenho e flexibilidade.
	
	\textbf{Palavras-chave}: microsserviços. \textit{REST}. \textit{gRPC}.
\end{resumo}